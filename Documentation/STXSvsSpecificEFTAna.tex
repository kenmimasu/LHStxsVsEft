\documentclass[11pt]{cernrep}
%\documentclass[12pt,letterpaper,twoside]{article}
%\documentclass[12pt,letterpaper]{article}
%\usepackage[pdftex]{graphicx,color}
\usepackage[pdftex]{graphicx,epsfig}
%\usepackage[dvips]{graphicx,color}
%\input{psfig}
%\input{epsf}
\usepackage{amsmath,amssymb}
%\usepackage[twoside,pdftex,letterpaper,text={6.5in,9in}]{geometry}
%\usepackage[twoside,dvips,letterpaper,text={6.5in,9in}]{geometry}
\usepackage[dvips,letterpaper,text={6.5in,9in}]{geometry}
\usepackage{fancyhdr}
\usepackage{verbatim}
\renewcommand{\baselinestretch}{1.1}
%\renewcommand{\theequation}{\thesection.\arabic{equation}}
%\numberwithin{equation}{section}

%       Symbol definitions
\newcommand\ltap{\
  \raise.3ex\hbox{$<$\kern-.75em\lower1ex\hbox{$\sim$}}\ }
\newcommand\gtap{\
  \raise.3ex\hbox{$>$\kern-.75em\lower1ex\hbox{$\sim$}}\ }
 \newcommand{\sss}{\scriptscriptstyle}
 \renewcommand{\phi}{\varphi}
%%%%%%%%%%%%%%%%%%%%%%%%%%%%%%%%%%%%%%%%%%%%%%%%%%%%%%%%%%%%%%%%%%%%%%%%

%  \simge and \simle make "approx greater than" and "approx less than"
\newcommand\simge{\mathrel{%
   \rlap{\raise 0.511ex \hbox{$>$}}{\lower 0.511ex \hbox{$\sim$}}}}
\newcommand\simle{\mathrel{
   \rlap{\raise 0.511ex \hbox{$<$}}{\lower 0.511ex \hbox{$\sim$}}}}

%  \slashcar puts a slash through a character to represent contraction
%  with Dirac matrices. Use \not instead for negation of relations, and use
%  \hbar for hbar.
\newcommand{\slashchar}[1]%
        {\kern .25em\raise.18ex\hbox{$/$}\kern-.75em #1}
%%%%%%%%%%%%%%%%%%%%%%%%%%%%%%%%%%%%%%%%%%%%%%%%%%%%%%%%%%%%%%%%%%%%%%%%%%
\def\lsim{\mathrel{\raise.3ex\hbox{$<$\kern-.75em\lower1ex\hbox{$\sim$}}}}
\def\gsim{\mathrel{\raise.3ex\hbox{$>$\kern-.75em\lower1ex\hbox{$\sim$}}}}
%%%%%%%%%%%%%%%%%%%%%%%%%%%%%%%%%%%%%%%%%%%%%%%%%%%%%%%%%%%%%%%%%%%%%%%%%%
\newcommand{\bs}{\boldsymbol}

\newenvironment{changemargin}[2]{\begin{list}{}{
        \setlength{\topsep}{0pt}\setlength{\leftmargin}{0pt}
        \setlength{\rightmargin}{0pt}
        \setlength{\listparindent}{\parindent}
        \setlength{\itemindent}{\parindent}
        \setlength{\parsep}{0pt plus 1pt}
        \addtolength{\leftmargin}{#1}\addtolength{\rightmargin}{#2}
        }\item }{\end{list}}
%
\begin{document}
\title{
%\vskip -15mm
Simplified Template Cross Sections: sensitivity to dimension-6 interactions at LHC run 2}
\author{Jorge~de~Blas$^{1,2}$,
Kristin~Lohwasser$^3$, Pasquale~Musella$^4$ and Ken~Mimasu$^5$}
\institute{$^1$Dipartimento di Fisica e Astronomia ``Galileo Galilei'', Universit\`a di Padova,\\ Via Marzolo 8, I-35131 Padova, Italy\\
$^2$INFN, Sezione di Padova, Via Marzolo 8, I-35131 Padova, Italy\\
$^3$Department of Physics and Astronomy, Sheffield University, Sheffield, UK \\
$^4$XXX\\
$^5$Centre for Cosmology, Particle Physics and Phenomenology (CP3), Universit\'e
catholique de Louvain, Chemin du Cyclotron, 2, B-1348 Louvain-la-Neuve, Belgium } 
\maketitle

\begin{abstract}
We perform a sensitivity study of the simplified template cross section (STXS) measurements to dimension-6 interactions within the standard model effective field theory framework. We focus on energy dependent effects in Higgs production in association with a $Z$-boson, $p p \to Z h \to \ell^+\ell^- b\bar{b}$. Several benchmark points are considered, with different values of a representative Wilson coefficient, alongside the Standard Model prediction as well as the dominant $Z+b\bar{b}$ background. We contrast the expected sensitivity obtained by the STXS to an optimised analysis exploiting multivariate techniques via a boosted decision tree classifier. The aim of this exercise is to estimate the amount information retained in the STXS binning, an therefore the power of the framework for model-independent hypothesis testing in Higgs physics. \textbf{what did we find?}
\end{abstract}

%%%%%%%%%%%%%%%%%%%%%%%%%%%%%%%%%
%%%%%%%%%%%%%%%%%%%%%%%%%%%%%%%%%

\newpage

\section{Introduction}
\label{sec:intro}
% All
The Standard Model Effective Field Theory (SMEFT) is, by now, a well established framework for parametrising new physics effects in the interactions of Standard Model (SM) particles in a model independent way. It has been and continues to be a key part of the LHC programme, complementary to direct searches for new physics. The framework employs an operator expansion in canonical dimension suppressed by a generic cutoff scale, $\Lambda$, assumed to be much larger than the electroweak (EW) scale, situated by the vacuum expectation value of the Higgs field, $v$. The SM Lagrangian is thus supplemented by higher dimension operators and truncated at dimension-6, the lowest dimension at which $B$- and $L$-number conserving operators appear.
\begin{align}
    \mathcal{L}=\mathcal{L}_{SM}+\sum_i \frac{c_i}{\Lambda^2}\mathcal{O}^i_{D=6}+\cdots,
\end{align} 
Where the dimension-5 Weinberg operator for neutrino masses has been neglected. New physics effects are then expected to appear scaling between $\sim v^2/\Lambda^2$ and a heightened energy dependence of $\sim E^2/\Lambda^2$.

One of the main strengths of the LHC in this respect is its ability to probe the high energy regime, in which it is expected that the sensitivity to EFT effects will be maximised to the large momentum flow through the modified  vertices accessing the stronger energy growth of the higher dimensional operators. Furthermore, the discovery of the Higgs boson in 2012 has opened a brand new avenue in constraining the SMEFT parameter space consisting of the various operators involving Higgs fields. Measurements of Higgs production and decay modes have already provided new constraints on many operators and have also helped to constrain some blind directions in existing fits to low-energy data such as precision electroweak measurements at LEP. 

In the first run of the LHC, a very successful programme of signal strength measurements took place, in which information from many searches was combined into a global fit to overall coupling modifiers between the Higgs and the rest of the SM particles. The natural evolution of these measurements for Run 2 is to subdivide the phase space and work towards differential observables in Higgs production and decay. To this end, a staged approach termed Simplified Template Cross Sections (STXS) is being developed~\cite{deFlorian:2016spz}, consisting of an increasingly fine-grained binning of kinematic observables, separated by production and decay mode. The aim is to provide measurements in mutually exclusive regions of phase space, performed in simplified fiducial volumes and unfolded to remove detector and acceptance effects. Ideally, these will be designed to maximise sensitivity to new physics.

Being one of the main elements of LHC searches for non-SM physics, it is of great interest to evaluate the sensitivity of the STXS measurements to SMEFT effects in Higgs boson interactions, particularly since they will be able to access these high energy tails of kinematic distributions. In particular, one would like to know how the information provided by a generic framework such as the STXS would compare to an optimised, dedicated search for SMEFT effects. Naively, one may expect some loss of information given, \emph{e.g.}, the finite binning of the distributions. In this study, we aim to quantify this difference by comparing and contrasting the ability to constrain SMEFT effects in Higgs production between the STXS measurements and an optimised analysis making use of multivariate methods to extract the maximum classification power of the SMEFT signals. We consider the concrete scenario of the (ZH) production of a Higgs boson decaying into a pair of $b$-quarks in association with a $Z$-boson  decaying to a pair of leptons, in the presence of a single EFT operator. We simulate several benchmark values for the operator Wilson coefficient within existing constraints from global fits along with the dominant reducible SM background and evaluate the statistical discriminating power of a hypothesis test using the STXS measurements versus a multivariate Boosted Decision Tree (BDT) classifier.

The paper is organised as follows. We first describe the Monte Carlo event generation procedure for the SM and EFT benchmarks in section~\ref{sec:gen}. Section~\ref{sec:tools} first describes the fiducial selection employed, the training and analysis implemented using the BDT classifier and the STXS binning used for ZH. In Section~\ref{sec:test}, we summarise the results of the selections and binning and perform a statistical hypothesis test to quantify the relative strengths of the two methods before concluding and laying out the avenues for further investigation in Section~\ref{sec:conclusions}

\section{Generated Models} 
\label{sec:gen}
%Ken
%!TEX root = /Users/Ken/Work/Projects/LesHouches2017/STXSvsEFT/repo/LHStxsVsEft/Documentation/STXSvsSpecificEFTAna.tex
The production of a Higgs boson in association with an EW gauge boson can be considered one of the canonical LHC processes sensitive to SMEFT effects. Operators which modify the Higgs coupling to these gauge bosons, introducing momentum dependent interactions modify the production rate, especially enhancing it in the high energy regime. The associate production process can naturally access this region of phase space since the Higgs is produced recoiling against the associated vector, meaning that the $p_T$ of the Higgs or vector boson are a faithful proxy for the energy flowing through the EFT vertex. Of the many dimension-6 operators that can contribute to this process, we consider 
\begin{align}
    \mathcal{O}_{\sss HW} &= 
    \frac{ig}{\Lambda^2} \big[D_\mu \phi^\dag T_{2k} D_\nu \phi\big] 
    W^{k,\mu \nu},
\end{align}
an operator from the so called strongly interacting light Higgs (SILH) basis~\cite{Giudice:2007fh,Contino:2013kra}. Here, $T_{2k}$ refers to the generators of $SU(2)_L$ normalised such that $\{T_{2i},T_{2j}\}=\delta_{ij}/2$ and the covariant derivative, $D_\mu$, for the Higgs field is defined as
\begin{align}
    D_\mu\phi &= \partial_\mu \phi -  i g T_{2k} W_\mu^k \phi - \frac12 i g' B_\mu \phi,
\end{align}
with $g$ and $g^\prime$ the weak and hypercharge gauge couplings respectively.



\section{Analysis}
\label{sec:tools}
\begin{itemize}
    \item Fiducial selection \& smearing
    \item subsection: BDT training methods \& classifiers
    \item subsection: STXS binning post Zbb classifier
\end{itemize}
%!TEX root = /Users/Ken/Work/Projects/LesHouches2017/STXSvsEFT/repo/LHStxsVsEft/Documentation/STXSvsSpecificEFTAna.tex
To begin the analysis on a realistic level, we first perform a simple fiducial selection on the event samples, to emulate a typical LHC selection that would be performed for the ZH process. To this end, we also implement a $p_T$ and $|\eta|$ dependent smearing function on the $b$-jet momenta to approximate finite detector resolution effects following the parametrised functions determined by the CMS particle-flow performance analysis~\cite{Sirunyan:2017ulk}.

Events are required to have two leptons satisfying $p_T > 25$ GeV and $|\eta|< 2.5$. Exactly two $b$-jets, as identified using truth-level information by {\sc MadAnalysis5}, are required satisfying $p_T > 20$ GeV and $|\eta|< 2.5$. We assume a flat $b$-tagging efficiency of $70\%$, corresponding to the DeepCSV medium working point defined in~\cite{Sirunyan:2017ezt}. Additionally, $Z$- and Higgs-boson mass windows are imposed on the invariant masses of the lepton and $b$-jet pairs such that $75 < M_{\ell\ell} < 105$ GeV and  $60 < M_{\ell\ell} < 140$ GeV. This defines our fiducial volume on which both the BDT training and STXS binning will be performed. Table~\ref{tab:FiducialXS} summarises the cross sections obtained after the fiducial selection for the Monte Carlo samples generated. The $H\to b\bar{b}$ branching fraction is computed in the SM and the SMEFT benchmark points using the e{\sc HDECAY}~\cite{Contino:2014aaa} interface of {\sc Rosetta}~\cite{Falkowski:2015wza} and folded into the cross section results.
\begin{table}
    \centering
    \begin{tabular}{|l|l|}
        \hline
        $pp\to b\,\bar{b}\,\ell^+\,\ell^-$& $\sigma_{\text{fid.}} [fb]$\tabularnewline
        \hline
        $ZH$ SM&5.55\tabularnewline
        $ZH$ $c_{\sss HW} = 0.03$&3.64\tabularnewline
        $ZH$ $c_{\sss HW} = -0.03$&2.21\tabularnewline
        $ZH$ $c_{\sss HW} = 0.01$&3.38\tabularnewline
        $ZH$ $c_{\sss HW} = -0.01$&2.50\tabularnewline
        $Z b\bar{b}$ SM&291.3\tabularnewline
        \hline
    \end{tabular}
    \caption{\label{tab:FiducialXS} Cross sections obtained at LO after imposing the fiducial selection cuts described in Section\ref{sec:tools}.}
\end{table}
Clearly, the $Z b\bar{b}$ background is overwhelmingly large even after the Higgs mass window selection. A realistic analysis will employ data driven methods to constrain this background using control regions. In the next section, one part of the BDT training is used to optimally reject this background in favour of the SM $ZH$ process.

%  

\subsection{STXS binning}

The generated samples for all signal benchmarks ($C_{HW}=???$) as well as the backgrounds (SM $pp \to ZH, H\to b\bar{b}$ and $pp\to Zb\bar{b}$) are categorized according to the STXS proposal for the $VH$ channel in Ref.~\cite{deFlorian:2016spz}. Within this framework, different regions of the phase space --referred to as ``bins'' for simplicity-- are defined, with the purpose of optimizing the sensitivity of the measurements while at the same time minimizing their dependence on theory assumptions. The different STXS bins are defined specifically for each Higgs production mode. For the process of interest
for the present exercise ($pp \to ZH \to \ell^+ \ell^- b\bar{b}$) the different stages of the categorization and the resulting bins are summarized as follows (see \cite{deFlorian:2016spz} for details):
%
\begin{itemize}
{\item  {\bf Stage 0:} Events with $\left|y_H\right| <2.5$ are selected.}
%
{\item {\bf Stage 1:} $ZH$ production is splitted into production via $q\bar{q}$ or $gg$ initial states. Our process sample was generated at LO and only contains $q\bar{q}\to ZH$ events. Events are subsequently classified according to the value of $p_T^Z$ and number of extra jets\footnote{JETS ARE DEFINED AS?} in the event as follows:
%
\vspace{0.25cm}
%
\begin{center}
${\bf \underline{q\bar{q}\to ZH}}$
\end{center}
\begin{eqnarray}
p_T^Z \in& [0,150]~\mathrm{GeV},& \nonumber\\
%
p_T^Z \in& [150,250]~\mathrm{GeV}&(0\mbox{-}\mathrm{j}),\\
%
p_T^Z \in& [150,250]~\mathrm{GeV}&(\geq 1\mbox{-}\mathrm{j}),\nonumber\\
%
p_T^Z >&250~\mathrm{GeV}.\nonumber&
\end{eqnarray}
}
%
%
{\item {\bf Stage 2:} On this last stage the low $p_T^Z$ bins are further separated according to the number of extra jets, while the high-$p_T^Z$ region is split at 400 GeV. The final set of STXS bins that apply in our case are the following six:
%
\vspace{0.25cm}
%
\begin{center}
${\bf \underline{q\bar{q}\to ZH}}$
\end{center}
\begin{eqnarray}
p_T^Z \in& [0,150]~\mathrm{GeV}&(0\mbox{-}\mathrm{j}),\nonumber\\
%
p_T^Z \in& [0,150]~\mathrm{GeV}&(\geq 1\mbox{-}\mathrm{j}),\nonumber\nonumber\\
%
p_T^Z \in& [150,250]~\mathrm{GeV}&(0\mbox{-}\mathrm{j}),\\
%
p_T^Z \in& [150,250]~\mathrm{GeV}&(\geq 1\mbox{-}\mathrm{j}),\nonumber\\
%
p_T^Z \in& [250,400]~\mathrm{GeV},&\nonumber\\
%
p_T^Z >&400~\mathrm{GeV}.\nonumber&
\end{eqnarray}
}
\end{itemize}
%
{[\bf EXPLAIN THE USE OF THE BDT VARIABLE TO REJECT BACKGR AFTER THE BDT SECTION IS DONE]}
{[\bf Figs: $S^2/B$ for the different benchmarks. Define S as new physics only?} 




\section{Statistical Hypothesis testing}
\label{sec:test}
%All!
\input{stats.tex}


\section{Conclusions}
\label{sec:conclusions}
\input{conclusions.tex}



\section*{Acknowledgments}

We thank the organizers and conveners of the Les Houches workshop, ``Physics
at TeV Colliders'', for a stimulating meeting. This work has received funding from the European Union's Horizon 2020 research and innovation programme / ERC Grant Agreement n. 715871. K. M. is supported in part by the Belgian Federal Science Policy Office through the Interuniversity Attraction Pole P7/37 and by the European Union’s Horizon 2020 research and innovation programme under the Marie Sk\l{}odowska-Curie grant agreement No. 707983.


%\vfil\eject


\bibliography{STXSvsSpecificEFTAna}
\bibliographystyle{utcaps}
%\bibliographystyle{lesHouches}
\end{document}
