%!TEX root = /Users/Ken/Work/Projects/LesHouches2017/STXSvsEFT/repo/LHStxsVsEft/Documentation/STXSvsSpecificEFTAna.tex
To begin the analysis on a realistic level, we first perform a simple fiducial selection on the event samples, to emulate a typical LHC selection that would be performed for the ZH process. To this end, we also implement a $p_T$ and $|\eta|$ dependent smearing function on the $b$-jet momenta to approximate finite detector resolution effects following the parametrised functions determined by the CMS particle-flow performance analysis~\cite{Sirunyan:2017ulk}.

Events are required to have two leptons satisfying $p_T > 25$ GeV and $|\eta|< 2.5$. Exactly two $b$-jets, as identified using truth-level information by {\sc MadAnalysis5}, are required satisfying $p_T > 20$ GeV and $|\eta|< 2.5$. We assume a flat $b$-tagging efficiency of $70\%$, corresponding to the DeepCSV medium working point defined in~\cite{Sirunyan:2017ezt}. Additionally, $Z$- and Higgs-boson mass windows are imposed on the invariant masses of the lepton and $b$-jet pairs such that $75 < M_{\ell\ell} < 105$ GeV and  $60 < M_{\ell\ell} < 140$ GeV. This defines our fiducial volume on which both the BDT training and STXS binning will be performed. Table~\ref{tab:FiducialXS} summarises the cross sections obtained after the fiducial selection for the Monte Carlo samples generated. The $H\to b\bar{b}$ branching fraction is computed in the SM and the SMEFT benchmark points using the e{\sc HDECAY}~\cite{Contino:2014aaa} interface of {\sc Rosetta}~\cite{Falkowski:2015wza} and folded into the cross section results.
\begin{table}
    \centering
    \begin{tabular}{|l|l|}
        \hline
        $pp\to b\,\bar{b}\,\ell^+\,\ell^-$& $\sigma_{\text{fid.}} [fb]$\tabularnewline
        \hline
        $ZH$ SM&5.55\tabularnewline
        $ZH$ $c_{\sss HW} = 0.03$&3.64\tabularnewline
        $ZH$ $c_{\sss HW} = -0.03$&2.21\tabularnewline
        $ZH$ $c_{\sss HW} = 0.01$&3.38\tabularnewline
        $ZH$ $c_{\sss HW} = -0.01$&2.50\tabularnewline
        $Z b\bar{b}$ SM&291.3\tabularnewline
        \hline
    \end{tabular}
    \caption{\label{tab:FiducialXS} Cross sections obtained at LO after imposing the fiducial selection cuts described in Section\ref{sec:tools}.}
\end{table}
Clearly, the $Z b\bar{b}$ background is overwhelmingly large even after the Higgs mass window selection. A realistic analysis will employ data driven methods to constrain this background using control regions. In the next section, one part of the BDT training is used to optimally reject this background in favour of the SM $ZH$ process.
